% Created 2020-01-16 Thu 15:49
% Intended LaTeX compiler: pdflatex
\documentclass[11pt]{article}
\usepackage[utf8]{inputenc}
\usepackage[T1]{fontenc}
\usepackage{graphicx}
\usepackage{grffile}
\usepackage{longtable}
\usepackage{wrapfig}
\usepackage{rotating}
\usepackage[normalem]{ulem}
\usepackage{amsmath}
\usepackage{textcomp}
\usepackage{amssymb}
\usepackage{capt-of}
\usepackage{hyperref}
\date{\today}
\title{Syllabus (FIN 302) Financial Management Principles}
\hypersetup{
 pdfauthor={},
 pdftitle={Syllabus (FIN 302) Financial Management Principles},
 pdfkeywords={},
 pdfsubject={},
 pdfcreator={Emacs 26.3 (Org mode 9.1.4)}, 
 pdflang={English}}
\begin{document}

\maketitle
\begin{center}
SUNY POLYTECHNIC INSTITUTE

SCHOOL OF BUSINESS ADMINISTRATION
\end{center}


\textbf{\textbf{Instructor}}: Matthew Brigida, Ph.D.

\textbf{\textbf{Office}}: Donovan 1277

\textbf{\textbf{Office Hours}}: Monday and Wednesday 2--3pm, and Tuesday 5:00--6:00

\textbf{\textbf{Email}}:  matthew.brigida@sunyit.edu

\textbf{\textbf{Class Location}}:  Donovan 1157

\textbf{\textbf{Class Day/Time}}: Tuesday, 6:00--9:30pm

\textbf{\textbf{Text}}: \href{https://www.amazon.com/dp/0073382469/ref=olp\_product\_details?\_encoding=UTF8\&me=}{Essentials of Corporate Finance, 7th ed, ISBN: 978-0073382463} | \href{https://www.amazon.com/gp/offer-listing/0073382469/ref=dp\_olp\_used?ie=UTF8\&condition=used}{Used Copies usually starting from about \$4} You also may use other editions of the text.

\textbf{\textbf{Supplementary Materials}}: \href{https://www.5minutefinance.org/}{5-Minute Finance} There may be additional topics in the \href{https://github.com/FinancialMarkets/5MinuteFinance}{Fundamentals and Corporate Finance Sections here}.


\section{Description}
\label{sec:org0faf15a}

\textbf{\textbf{Course Catalog Description}}: General principles of corporate/managerial finance are presented. Topics include the tax environment, an overview of financial planning and control, working capital management, long term financing, time value of money, capital budgeting, cost of capital, dividend policy, agency theory, and international financial issues. Responsibilities and functions performed by financial analysts, financial managers, and chief financial managers are discussed. (Pre-requisite: ACC 201 or equivalent.)

\subsection{Overview}
\label{sec:orgb6a5fb4}

\begin{itemize}
\item This course will present the fundamentals of managerial finance as a vital part of the complete business ecosystem. Importance is placed on becoming familiar with the rudimentary tools and techniques that act as the basis for all further study and function of finance.

\item Emphasis is also placed on exploring the human, technological, and environmental impacts on financial decision making and the societal variables that facilitate positive (financial and social) outcomes.

\item Attention to current events and linkage between financial markets and corporate decision-making and vis-a-vis is discussed.

\item Introduction to the role of the change agent and changing technology in financial analysis and financial decision-making processes.
\end{itemize}

\subsection{Course Learning Outcomes \& Objectives}
\label{sec:orge0ce7f2}

CLO 1. Technical Competence: Adept in applying technology to solve institutional problems and enable effective financial decision making.

CLO 2. Analytical Problem Framing: Demonstrate individual capacity to evaluate and deploy analytical methods selected from a diverse portfolio of tools analyze and manage common financial decisions.

CLO 3. Strategic and Integrative Thinking: Understand the baseline resources available for analyzing and managing a firm’s financial performance. Including collecting data, processing information and evaluating and communicating outcomes with partners; differentiate between the accounting function as a preparer of data and information and the finance function as a user of information for decision making and the role of ethics in the process.

CLO 4. Leadership and Communication: Be capable of expressing key financial concepts and terms commonly used in the field; by using effective written, oral and interpersonal communications to contribute to the financial performance of domestic firms, global organizations and other international relationships.

\section{Exams}
\label{sec:org0d59cc0}

There will be three exams (two during the semester and a final).  The exams will be multiple choice.

\section{Project}
\label{sec:orga5dce80}

In groups, students will either:

\begin{enumerate}
\item Value a tolling agreement on a natural gas fired power plant near Boston Mass via discounted cash flows.  This project will make heavy use of core concepts learned in this course---namely the calculation of cash flows and the time-value-of-money.  \href{https://www.eia.gov/electricity/wholesale/\#history}{Data are here}.
\end{enumerate}

or 

\begin{enumerate}
\item Estimate the relationship between capital levels and measures of bank income and performance.
\end{enumerate}

In addition, students will gain experience handling data, and making reasonable assumptions necessary in real-world analyses.

\section{Attendance/Quizzes/Participation}
\label{sec:org9bb0eb0}

Throughout the semester I will take attendance, give unanounced quizzes, and otherwise evaluate your participation.  Failure to attend class and participate will reduce your participation score, unless your absence is due to a \textbf{\textbf{verifiable}} medical or family emergency.  In such a case you must provide documentation.

\section{Grading}
\label{sec:orgff371bc}

\begin{center}


\begin{center}
\begin{tabular}{lr}
Item & Points\\
\hline
Exam 1 & 20\\
Exam 2 & 20\\
Final Exam & 25\\
Project & 15\\
Attendance/Pop Quizzes/Other Participation & 20\\
\hline
Total Points & 100\\
\hline
\end{tabular}
\end{center}
\end{center}

\begin{center}
\textbf{\textbf{Final grades will be assigned according to the following scale}}:
\end{center}

\begin{itemize}
\item 90 - 100 A
\item 80 - 89.9 B
\item 70 - 79.9 C
\item 60 - 69.9 D
\item \(<\) 60 F
\end{itemize}

\begin{quote}
+/- grades may be assigned at the instructors discretion.
\end{quote}

\subsection{An Important Note on Grading}
\label{sec:org1ec0cfc}

\begin{quote}
There is no special consideration if you need a certain grade in this course to graduate.  \textbf{\textbf{If you require a certain grade in this class to graduate it is your responsibility to earn that grade.}} Specifically if you receive a `D` in this course I will not allow you to do extra assignments after the course is complete in exchange for a higher grade. 
\end{quote}


\section{Guidelines and Accommodations}
\label{sec:org9a5160b}

Academic Integrity Policy Students Enrolled in this course are required to understand and fully comply with all aspects of the Academic Integrity Policy as described in the SUNY Polytechnic Institute Handbook (available at:  \url{https://sunypoly.edu/pdf/student\_handbook.pdf} )

\subsection{Accommodations for Students with Disabilities}
\label{sec:org043c9dd}

Students with disabilities are welcome at SUNY Polytechnic Institute.
The Disabilities Services Office is located in the Career Services Suite, B101, Kunsela Hall
Hours: Monday through Friday 8:30 a.m. – 4:30 p.m. or by appointment.
E-mail: suzanne.sprague@sunyit.edu
Phone: (315) 792-7170

Any current or prospective student may contact our office to discuss potential academic accommodations. Typical accommodations include extended time for testing, testing in a quiet location, textbooks in alternate format, and others as determined by the nature of the disability. These accommodations must be supported by documentation from outside sources, such as a recent psychological evaluation or medical report that clearly identifies the nature of your disability and the impact of your disability or treatment upon learning. (SUNY Poly is not responsible for providing evaluation or funding to complete the needed documentation.)

The Disabilities Services Office will assist with requesting the required documentation or exploring resources that may provide testing or documentation. Once documentation is received, the Disability Counselor meets with the student to discuss the information based on his or her experiences and perspective. A student’s explanation of how his or her disability affects learning and what accommodations are needed is extremely important. Once a determination is made regarding the reasonable and appropriate accommodations, a plan is written which students sign and share with instructors. This plan does not disclose the nature of the disability, although many students have found that discussing their circumstances with faculty can be helpful.

Accommodations are implemented to ensure compliance with the Americans with Disabilities Act of 1990 (ADA) and Section 504 (subsection E) of the Rehabilitation Act. The intent of which is to provide access for otherwise qualified persons. SUNY Poly is not required to lower or substantially modify essential academic requirements, or make modifications that would fundamentally alter the nature of a service, program, or activity or that would result in an undue financial or administrative burden. Additionally, accommodation plans must be updated each semester.

The Disabilities Services Office is happy to advocate or coordinate with outside service providers, by student request and with written consent to communicate. Parents should be aware that legal rights and responsibilities change from high school to the college, both in terms of the type of support provided and in terms of parental involvement in the process. Students are solely responsible for self-identifying and following up with our office for any needs that they may have. Any student who wishes to have our office communicate with a parent must sign a written consent for permission to communicate.

\subsection{Course Syllabus Disclosure Statement Spring 2020}
\label{sec:org34bbec6}

\begin{quote}
Accommodations for Students with Disabilities

In compliance with the Americans with Disabilities Act of 1990 and Section 504 of the Rehabilitation Act, SUNY Polytechnic Institute is committed to ensuring comprehensive educational access and accommodations for all registered students seeking access to meet course requirements and fully participate in programs and activities.  Students with documented disabilities or medical conditions are encouraged to request these services by registering with the Office of Disability Services.  For information related to these services or to schedule an appointment, please contact the Office of Disability Services using the information provided below.

Evelyn Lester, Director
Office of Disability Services
lestere@sunypoly.edu
(315) 792-7170

Utica Campus
Peter J. Cayan Library, L145

Albany Campus
Suite 309, Students Services Office
NanoFab South
\end{quote}

\section{Tentative Outline by Week}
\label{sec:org0f1c385}

\begin{itemize}
\item 1/21: Chapter 1 \& 2
\item 1/28:  Chapter 3 and 4
\item 2/4: Chapter  5
\item 2/11: Chapter 6
\item 2/18:
\item 2/25: Chapter 7
\item 3/3: Chapter 8
\item 3/10: Chapter 9
\item 3/17: Spring Break
\item 3/24: Chapter 10
\item 4/1:  Chapter 11 \& Chapter 12
\item 4/7:
\item 4/15: Chapter 13 and 14
\item 4/22: Chapter 15
\item 4/29: exam review
\end{itemize}
\end{document}